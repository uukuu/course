\chapter{实验 2: 线性表实现及应用}
%\addcontentsline{toc}{chapter}{实验 1: 渐进分析和排序算法}
\section{Task 1: 为指定的 List ADT 实现各种数据结构}

\subsection{代码实现}

\subsubsection{顺序数组}

\begin{lstlisting}
template<typename T>
class List_ADT{
	int cursor, len;
	vector<T> val;
	public:
	void init(int length)
	{
		val.resize(length);
		cursor = -1;
		len = 0;
		return;
	}
	
	void insert(T newElement)
	{
		if(isFull())
		{
			printf("Error! Out of the list capacity!\n");
			return;
		}
		for(int i = len;i > cursor + 1; i--)
		val[i] = val[i - 1];
		len ++;
		val[++ cursor] = newElement;
	}
	void remove()
	{
		if(isEmpty()) return;
		len --;
		for(int i = cursor; i < len; i ++)
		val[i] = val[i + 1];
		cursor = cursor == len ? 0 : cursor;
		if(len == 0)
		cursor = -1;
		return;
	}
	void replace(T newElement)
	{
		if(isEmpty()) return;
		val[cursor] =  newElement;
		return;
	}
	void clear()
	{
		len = 0;
		cursor = -1;
		return;
	}
	bool isEmpty()
	{
		return len == 0;
	}
	bool isFull()
	{
		return len == (int)val.size();
	}
	bool gotoBeginning()
	{
		if(isEmpty()) return false;
		else
		{
			cursor = 0;
			return true;
		}
	}
	bool gotoEnd()
	{
		if(isEmpty()) return false;
		else
		{
			cursor = len - 1;
			return true;
		}
	}
	bool gotoNext()
	{
		if(cursor + 1 != len)
		{
			cursor ++;
			return true;
		}
		else return false;
	}
	bool gotoPrev()
	{
		if(isEmpty()) return false;
		else if(cursor == 0) return false;
		else
		{
			cursor --;
			return true;
		}
	}
	T getCursor()
	{
		if(isEmpty()) return 0;
		return val[cursor];
	}
	void showStructure()
	{
		if(isEmpty()) printf("Empty list ");
		else
		{
			for(int i=0;i<len;i++) 
			putchar(val[i]),putchar(' ');
		}
		printf("{capacity = %d, length = %d, cursor = %d}\n",(int)val.size(),len,cursor);
	}
	void moveToNth(int n)
	{
		T tmp = getCursor();
		remove();
		cursor = n - 1;
		insert(tmp);
	}
	bool find(T searchElement)
	{
		while(getCursor() != searchElement&&gotoNext());
		return getCursor() == searchElement;
	}
};
char s[100010];
int main()
{
	freopen("list_testcase.txt","r",stdin);
	freopen("数组.txt","w",stdout);
	List_ADT<char> a;
	a.init(512);
	while(gets(s) != NULL)
	{
		int n = strlen(s);
		for(int i = 0; i < n; i ++)
		{
			if(s[i] == ' ')	continue;
			if(s[i] == '+')
				a.insert(s[++i]);
			else if(s[i] == '-')
				a.remove();
			else if(s[i] == '=')
				a.replace(s[++ i]);
			else if(s[i] == '#')
				a.gotoBeginning();
			else if(s[i] == '*')
				a.gotoEnd();
			else if(s[i] == '>')
				a.gotoNext();
			else if(s[i] == '<')
				a.gotoPrev();
			else if(s[i] == '~')
				a.clear();
		}
		a.showStructure();
	}
	return 0;
}

\end{lstlisting}

\subsubsection{单向链表}

主函数同上, 此处不多赘述.

\begin{lstlisting}
template<typename T>
class List_ADT{
	int cursor, len, lenLimit;
	struct Point{
		T val;
		Point *next;
	};
	Point *head = nullptr, *p = nullptr, *tmp = nullptr;
	public:
	void init(int length)
	{
		head = nullptr;
		cursor = -1;
		len = 0;
		lenLimit = length;
		return;
	}
	
	void insert(T newElement)
	{
		if(isFull())
		{
			printf("Error! Out of the list capacity!\n");
			return;
		}
		if(len == 0)
		{
			head = p = new Point;
			p -> next = nullptr;
			p -> val = newElement;
		}
		else
		{
			tmp = p -> next;
			p -> next = new Point;
			p = p -> next;
			p -> val = newElement;
			p -> next = tmp;
		}
		len ++;
		cursor ++;
		return;
	}
	
	void remove()
	{
		if(isEmpty()) return;
		if(p == head)
		{
			p = head -> next;
			free(head);
			head = p;
		}
		else
		{
			tmp = p;
			gotoPrev();
			p -> next = tmp -> next;
			free(tmp);
			p = p -> next;
			cursor ++;
			if(p == nullptr) p = head, cursor = 0;
		}
		len --;
		if(len == 0)
		{
			cursor = -1;
			head = p = nullptr;
		}
		return;
	}
	void replace(T newElement)
	{
		if(isEmpty()) return;
		p->val = newElement;
		return;
	}
	void clear()
	{
		len = 0;
		cursor = -1;
		p = head;
		while(p!= nullptr)
		{
			tmp = p;
			p = p -> next;
			free(tmp);
		}
		head = p = nullptr;
		return;
	}
	
	bool isEmpty()
	{
		return len == 0;
	}
	
	bool isFull()
	{
		return len == lenLimit;
	}
	
	bool gotoBeginning()
	{
		if(isEmpty()) return false;
		else
		{
			cursor = 0;
			p = head;
			return true;
		}
	}
	
	bool gotoEnd()
	{
		if(isEmpty()) return false;
		else
		{
			gotoBeginning();
			while(gotoNext());
			return true;
		}
	}
	
	bool gotoNext()
	{
		if(isEmpty()) return false;
		if(p -> next != nullptr)
		{
			cursor ++;
			p = p -> next;
			return true;
		}
		else return false;
	}
	bool gotoPrev()
	{
		if(isEmpty()) return false;
		if(p == head) return false;
		tmp = p;
		gotoBeginning();
		while(p -> next != tmp) gotoNext();
		return true; 
	}
	T getCursor()
	{
		if(isEmpty()) return nullptr;
		return p -> val;
	}
	void showStructure()
	{
		if(isEmpty()) printf("Empty list ");
		else
		{
			tmp = head;
			while(tmp != nullptr)
			{
				putchar(tmp -> val);
				putchar(' ');
				tmp = tmp -> next;
			}
		}
		printf("{capacity = %d, length = %d, cursor = %d}\n",lenLimit,len,cursor);
	}
	void moveToNth(int n)
	{
		T tmp = getCursor();
		remove();
		gotoBeginning();
		for(int i = 1; i < n; i ++) 
		gotoNext();
		insert(tmp);
	}
	bool find(T searchElement)
	{
		while(getCursor() != searchElement && gotoNext());
		return getCursor() == searchElement;
	}
};
\end{lstlisting}

\subsubsection{双向链表}

\begin{lstlisting}
template<typename T>
class List_ADT{
	int cursor, len, lenLimit;
	struct Point{
		T val;
		Point *next,*pre;
	};
	Point *head = nullptr, *p = nullptr, *tmp = nullptr;
	public:
	void init(int length)
	{
		head = p = nullptr;
		cursor = -1;
		len = 0;
		lenLimit = length;
		return;
	}
	
	void insert(T newElement)
	{
		if(isFull())
		{
			printf("Error! Out of the list capacity!\n");
			return;
		}
		if(len == 0)
		{
			head = p = new Point;
			p -> next = p -> pre = nullptr;
			p -> val = newElement;
		}
		else
		{
			tmp = p -> next;
			p -> next = new Point;
			p -> next -> pre = p;
			p = p -> next;
			p -> val = newElement;
			p -> next = tmp;
			if(tmp != nullptr) 
			tmp -> pre = p;
		}
		len ++;
		cursor ++;
		return;
	}
	
	void remove()
	{
		if(isEmpty()) return;
		if(p == head)
		{
			p = head -> next;
			free(head);
			head = p;
			if(p != nullptr)
			p -> pre = nullptr;
		}
		else
		{
			tmp = p;
			p -> pre -> next = p -> next;
			if(p -> next != nullptr)
			p -> next -> pre = p -> pre;
			p = p -> next;
			free(tmp);
			if(p == nullptr) p = head, cursor = 0;
		}
		len --;
		if(len == 0)
		{
			cursor = -1;
			head = p = nullptr;
		}
		return;
	}
	void replace(T newElement)
	{
		if(isEmpty()) return;
		p->val = newElement;
		return;
	}
	void clear()
	{
		len = 0;
		cursor = -1;
		p = head;
		while(p!= nullptr)
		{
			tmp = p;
			p = p -> next;
			free(tmp);
		}
		head = p = nullptr;
		return;
	}
	
	bool isEmpty()
	{
		return len == 0;
	}
	
	bool isFull()
	{
		return len == lenLimit;
	}
	
	bool gotoBeginning()
	{
		if(isEmpty()) return false;
		else
		{
			cursor = 0;
			p = head;
			return true;
		}
	}
	
	bool gotoEnd()
	{
		if(isEmpty()) return false;
		else
		{
			while(gotoNext());
			return true;
		}
	}
	
	bool gotoNext()
	{
		if(isEmpty()) return false;
		if(p -> next != nullptr)
		{
			cursor ++;
			p = p -> next;
			return true;
		}
		else return false;
	}
	bool gotoPrev()
	{
		if(isEmpty()) return false;
		if(p == head) return false;
		p = p -> pre;
		cursor --;
		return true; 
	}
	T getCursor()
	{
		if(isEmpty()) return nullptr;
		return p -> val;
	}
	void showStructure()
	{
		if(isEmpty()) printf("Empty list ");
		else
		{
			tmp = head;
			while(tmp != nullptr)
			{
				putchar(tmp -> val);
				putchar(' ');
				tmp = tmp -> next;
			}
		}
		printf("{capacity = %d, length = %d, cursor = %d}\n",lenLimit,len,cursor);
	}
	void moveToNth(int n)
	{
		T tmp = getCursor();
		remove();
		gotoBeginning();
		for(int i = 1; i < n; i ++) 
		gotoNext();
		insert(tmp);
	}
	bool find(T searchElement)
	{
		while(getCursor() != searchElement && gotoNext());
		return getCursor() == searchElement;
	}
};
\end{lstlisting}

以上代码均通过实验中提供的测试用例, 利用系统自带的 $\t{fc}$ 比较上述代码输出文件于实验下发文件, 均返回无差异.

对于实验题目注 $\t{List}$ 实现的存储方式为链式结构, 其 $\t{capacity}$ 应为真实结点个数, 其实就应该等于输出中的 $\t{len}$ 的值, 所以在测试链表时, 只需在 $\t{capacity}$ 处输出 $512$ 并直接与结果比较即可.

\subsection{代码差异分析}

仔细研究各函数的写法, 我们可以找到一些“本质”的函数, 此处认为与存储方式有关的函数是“本质”的. 比如 $\t{moveToNth,find,gotoEnd}$ 这些函数就是非“本质”的. 

也就是说这些函数的实现和存储方式无关, 可以由其它函数多次调用实现, 对于上述三种存储表示其实现的代码是相同的. 

而为了简化代码的实现, 我们希望“本质”的函数尽量少, 这样在换用不同的存储逻辑实现相同的操作时, 我们需要对代码的改动就少.

更进一步的, 单向链表和双向链表的差异很小, 对于这两种存储逻辑, 需要修改的部分其实只有 $\t{gotoPrev}$.

\subsection{不同存储方式复杂度分析}
$$
\begin{array}{|c|c|c|c|}\hline
	\t{函数名称} & \t{顺序数组} & \t{单向链表} & \t{双向链表} \\\hline
	\t{insert}  & \t O(n) & \t O(1) & \t O(1) \\\hline
	\t{remove}  & \t O(n) & \t O(n) & \t O(1) \\\hline
	\t{replace} & \t O(1) & \t O(1) & \t O(1) \\\hline
	\t{clear}   & \t O(1) & \t O(n) & \t O(n) \\\hline
	\t{isEmpty} & \t O(1) & \t O(1) & \t O(1) \\\hline
	\t{isFull}  & \t O(1) & \t O(1) & \t O(1) \\\hline
	\t{gotoBeginning}  & \t O(1) & \t O(1) & \t O(1) \\\hline
	\t{gotoEnd} & \t O(1) & \t O(n) & \t O(n) \\\hline
	\t{gotoNext}& \t O(1) & \t O(1) & \t O(1) \\\hline
	\t{gotoPrev}& \t O(1) & \t O(n) & \t O(1) \\\hline
	\t{getCursor}  & \t O(1) & \t O(1) & \t O(1) \\\hline
	\t{showStructure}  & \t O(n) & \t O(n) & \t O(n) \\\hline
	\t{moveToNth}  & \t O(n) & \t O(n) & \t O(n) \\\hline
	\t{find}    & \t O(n) & \t O(n) & \t O(n) \\\hline
\end{array}
$$

一些小细节及分析:

\begin{itemize}
	\item[(1)] 对于 $\t{clear}$ 的复杂度, 此处考虑到要将单向/双向链表新建的结点空间释放, 需要遍历整个链表所以其复杂度为 $\t O(n)$. 相应的, 对于数组的清空, 我们只需清楚光标和长度即可, 因为数组的空间是在最初始申请的, 不会随着程序运行而变化, 而且每次操作都是赋值, 不清空也不会对后续操作产生影响.
	\item[(2)] 对于 $\t{moveTonNth}$ 操作, 虽然三种存储用时均为 $\t O(n)$, 但其具体的原因却不相同, 数组是因为插入删除操作需要 $\O(n)$ 的时间, 寻找第 $n$ 是 $\t O(1)$, 所以该操作用时为 $\t O(n)$. 而对于双向链表, 其主要复杂度用于寻找第 $n$ 个位置, 而相应的插入删除操作则仅仅为 $\t O(1)$.
	\item[(3)] 对于单向链表和双向链表, 其本质差距只有 $\t{gotoPrev}$ 这个函数的时间消耗, 双向链表可以 $\t O(1)$ 查询前驱, 进而在 $\t O(1)$ 的复杂度完成删除操作.
	\item[(4)] 如果我们在链表处理时在动态维护一个尾指针, 那么我们可以将 $\t{gotoEnd}$ 的复杂度优化至 $\t O(1)$, 但不会使其他操作用时间增加.
	\item[(5)] 如果进行 (4) 的优化, 仅通过上述表格, 我们认为数组远不如链表, 因为链表的每一项操作都优于数组. 但实际上正如 (2) 所述, 数组的优势其实是可以在 $\t O(1)$ 的时间内随机访问, 而相应的链表则需要 $\t O(n)$ 的时间, 而这种访问操作并没有体现在上述测试过程中. 并且该操作在很多实际应用中占了大部分, 所以数组在这方面优势很大. 这两类存储应该说各有优劣, 要根据不同的应用场景进行选择.
\end{itemize}

\section{Task 2: 为指定的 DQueue ADT 实现两种数据结构}

\subsection{代码实现}

\subsubsection{顺序数组}

利用循环数组实现.

\begin{lstlisting}
template<typename T>
class DQueue{
	vector<T>q;
	int siz, head, tail, sizeLimit;
	T tmp;
	public:
	void init(int length)
	{
		q.resize(length);
		sizeLimit = length;
		siz = 0; 
		head = 0;
		tail = -1;
	}
	bool isEmpty()
	{
		return siz == 0;
	}
	bool isFull()
	{
		return siz == sizeLimit;
	}
	void clear()
	{
		siz = 0;
		head = 0;
		tail = -1;
	}
	
	int size()
	{
		return siz;
	}
	void enqueueToRear(T element)
	{
		if(isFull()) return;
		tail = (tail + 1) % sizeLimit;
		q[tail] = element;
		++ siz;
		return;
	}
	void enqueueToFront(T element)
	{
		if(isFull()) return;
		head = (head + sizeLimit - 1) % sizeLimit;
		q[head] = element;
		++ siz;
		return;
	}
	T dequeueFromFront()
	{
		if(isEmpty()) return {};
		tmp = q[head];
		head = (head + 1) % sizeLimit;
		-- siz;
		return tmp;
	}
	T dequeueFromRear()
	{
		if(isEmpty()) return {};
		tmp = q[tail];
		tail = (tail + sizeLimit - 1) % sizeLimit;
		-- siz;
		return tmp;
	}
	T getFront()
	{
		if(isEmpty()) return {}; 
		return q[head];
	}
	T getRear()
	{
		if(isEmpty()) return {}; 
		return q[tail];
	}
};
\end{lstlisting}

\subsubsection{双向链表}

\begin{lstlisting}
template<typename T>
class DQueue{
	struct Point{
		T val;
		Point *next, *pre;
	};
	int siz, sizeLimit;
	T tmpval;
	Point *head, *tail, *tmp;
	public:
	void init(int length)
	{
		sizeLimit = length;
		siz = 0;
		head = tail = nullptr;
	}
	void clear()
	{
		while(head != nullptr)
		{
			tmp = head;
			head = head -> next;
			free(tmp);
		}
		siz = 0;
		head = tail = nullptr;
	}
	bool isEmpty()
	{
		return siz == 0;
	}
	bool isFull()
	{
		return siz == sizeLimit;
	}
	int size()
	{
		return siz;
	}
	void enqueueToRear(T element)
	{
		if(isFull()) return;
		if(isEmpty())
		{
			tail = head = new Point;
			head -> val = element;
			head -> pre = head -> next = nullptr;
		}
		else
		{
			tail -> next = new Point;
			tail -> next -> pre = tail;
			tail = tail -> next;
			tail -> val = element;
			tail -> next = nullptr;
		}
		++ siz;
		return;
	}
	void enqueueToFront(T element)
	{
		if(isFull()) return;
		if(isEmpty())
		{
			tail = head = new Point;
			head -> val = element;
			head -> pre = head -> next = nullptr;
		}
		else
		{
			head -> pre = new Point;
			head -> pre -> next = head;
			head = head -> pre;
			head -> val = element;
			head -> pre = nullptr;
		}
		++ siz;
		return;
	}
	T dequeueFromFront()
	{
		if(isEmpty()) return {};
		tmp = head;
		tmpval = tmp -> val;
		head = head -> next;
		free(tmp);
		-- siz;
		return tmpval;
	}
	T dequeueFromRear()
	{
		if(isEmpty()) return {};
		tmp = tail;
		tmpval = tmp -> val;
		tail = tail -> pre;
		free(tmp);
		-- siz;
		return tmpval;
	}
	T getFront()
	{
		if(isEmpty()) return {}; 
		return head -> val;
	}
	T getRear()
	{
		if(isEmpty()) return {}; 
		return tail -> val;
	}
};
\end{lstlisting}

\subsection{滑动窗口问题}

\subsubsection{问题分析}

此问题可以采用单调队列这一算法完成. 考虑到靠后的较大值, 肯定比前面的较小值更优, 因为再取最大值的时候, 当靠后的值更大时, 我们可以直接忽略前面比它小的值.

基于此思想, 我们区别于直接入队, 而是先把之前队列中比即将要插入的值小的值从队尾弹出. 然后再将当前值从队尾插入. 利用归纳的思想, 不难证明在上述操作过程中, 我们的这个队列始终保持一个单调性. 即从队首到队尾的值是单调递减的.

而对于滑动窗口这个问题, 我们还需要记录队列中每个值的位置, 而当队首元素的位置超出当前计算的区间时, 就需要将其从队首弹出.

由此, 我们就需要之前实现过的双端队列 $\t{DQueue}$ 数据结构来完成这些操作.

而对于存储位置, 我们有两种操作, 一种是直接在 $\t{DQueue}$ 中放入自定义结构体类型, 而每个结构体变量都包含值及其位置, 另一种是, 在 $\t{DQueue}$ 中放入位置, 然后每次根据位置去调取该位置上的值再进行比较.

\subsubsection{代码实现}
以下代码中的 $\t{DQueue}$ 类, 就是上述列举的两个代码, 并且下述代码均通过了洛谷 $\t{P1886}$  滑动窗口 /【模板】单调队列 题目. 第一行输出为每个区间的最小值, 第二行输出为最大值.

\begin{lstlisting}
struct node{
	int pos, val;
};
int n, k;
vector<int>a;
int main()
{
	read(n), read(k);
	DQueue<node> q;
	a.resize(n + 2);
	q.init(n + 2); 
	for(int i = 1; i <= n; i ++)
	{
		read(a[i]);
		while(!q.isEmpty() && i - q.getFront().pos >= k) q.dequeueFromFront();
		while(!q.isEmpty() && q.getRear().val >= a[i]) q.dequeueFromRear();
		q.enqueueToRear({i,a[i]});
		if(i >= k) write(q.getFront().val, i != n ? ' ' : '\n'); 
	}
	q.clear();
	for(int i = 1; i <= n; i ++)
	{
		while(!q.isEmpty() && i - q.getFront().pos >= k) q.dequeueFromFront();
		while(!q.isEmpty() && q.getRear().val <= a[i]) q.dequeueFromRear();
		q.enqueueToRear({i,a[i]});
		if(i >= k) write(q.getFront().val, i != n ? ' ' : '\n');
	}
	flushout();
	return 0;
}
\end{lstlisting}

\section{Task 3: 栈}

\subsection{快排非递归转化}

\subsubsection{问题分析}

要进行非递归转化, 我们需要关注快速排序在递归过程中哪些值被传递到了下一层.

发现, 快速排序实际上只用传输区间端点 $l,r$ 这两个值, 所以我们可以利用栈, 每一层就存放这两个值, 然后在处理完当层之后, 将下一层递归的区间加入栈即可.

\subsubsection{代码实现}

\begin{lstlisting}
int part(vector<int>&num,int l,int r)
{
	if(l == r) return -1;
	int pos = random(l,r);
	int p = num[pos];
	swap(num[pos],num[r]);
	int i = l, j = r - 1;
	while(i < j)
	{
		while(i < j&& num[i] < p) i ++;
		while(i < j&& num[j] >= p) j --;
		swap(num[i], num[j]);
	}
	if(num[i] >= num[r]) swap(num[i], num[r]);
	else i ++;
	return i;
}
void Quicksort(vector<int>&num)
{
	stack<pii> stk;
	stk.push({0, num.size() - 1});
	while(stk.size())
	{
		pii p = stk.top();
		stk.pop();
		int i = part(num,p.first,p.second);
		if(i == -1)continue;
		if(p.first < i - 1) stk.push({p.first, i - 1});
		if(i + 1 < p.second) stk.push({i + 1, p.second});
	}
	return;
}
\end{lstlisting} 

\subsection{算术混合运算表达式计算}

\subsubsection{问题分析及算法设计}

利用栈, 将中缀表达式转换为后缀表达式(逆波兰表达式), 并进行计算.

由于本题只要求最终结果, 所以并未求出完整的后缀表达式, 而是在计算过程中同步计算.

先不考虑括号.

具体思路, 考虑给每种运算符赋予一个优先级, 我们用两个栈分别维护运算符和数值, 而当之前压入栈的运算符优先级小于等于当前运算符时, 就将之前的运算符出栈并进行计算. 计算后再将当前的运算符压入栈.

而当多了括号也很简单, 在遇到左括号的时候, 我们只需将其入栈, 不需要任何额外操作. 当遇到右括号时, 我们一直做出栈操作, 直至栈顶为左括号.

至于出栈操作, 具体点就是取出运算符的栈顶和数值栈顶两个元素, 将这两个元素做相应的运算. 然后再将结果压入数值栈.

如此操作之后, 当最终运算符栈为空, 数值栈只剩一个元素时, 这个元素就是表达式的结果.

注: 一些细节, 当 '-' 号前为空或者是左括号时, 此时 '-' 号应理解为负号, 而不是减号. 需要特殊处理, 为了统一, 实际上此时只需在数值栈压入一个 $0$, 从而将 '-' 号视作减号即可.

\subsubsection{代码实现}

\begin{lstlisting}
const int N = 2e5 + 10;
char s[N];
int n, stk_p[N],top_p,top_val;
double stk_val[N];
int pr(char opt)
{
	switch(opt)
	{
		case '+':
		return 1;
		case '-':
		return 1;
		case '*':
		return 2;
		case '/':
		return 2;
		case '^':
		return 3;		
	}
	return 0;
}
double work(int opt, double v1, double v2)
{
	switch(opt)
	{
		case '+':
		return v1 + v2;
		case '-':
		return v1 - v2;
		case '*':
		return v1 * v2;
		case '/':
		return v1 / v2;
		case '^':
		return pow(v1,v2);
	}
	return 0;
}
void pop()
{
	double tmp = work(stk_p[top_p], stk_val[top_val - 1], stk_val[top_val]);
	top_val -= 2;
	stk_val[++top_val] = tmp;
	top_p --;
	return;
}
int main()
{
	scanf("%s", s);
	n = strlen(s);
	if(s[0] == '-') stk_val[++ top_val] = 0;
	for(int i = 0; i < n; i ++)
	{
		if(s[i] == '.') continue;
		if(s[i] >= '0' && s[i] <= '9')
		{ 
			double res = 0;
			while(s[i + 1] >= '0' && s[i + 1] <= '9')
			res = res * 10 + s[i] - '0', i++;
			res = res * 10 + s[i] - '0';
			if(s[i + 1] == '.' && s[i + 2] >= '0' && s[i + 2] <= '9')
			{
				double r = 0, p = 10;
				i += 2;
				while(s[i + 1] >= '0' && s[i + 1] <= '9')
				r += (s[i] - '0') / p, p *= 10;
				r += (s[i] - '0') / p;
				res += r;
			}
			stk_val[++top_val] = res;
		}
		else if(s[i] == '(') 
		{
			stk_p[++ top_p] = '(';
			if(s[i + 1] == '-') stk_val[++ top_val] = 0;
		}
		else if(s[i] == ')')
		{
			while(top_p && stk_p[top_p] != '(') pop();
			if(top_p == 0)
			{
				puts("括号不匹配,缺少左括号");
				return 0;
			}
			top_p --;
			
		}
		else if(pr(s[i]) > 0)
		{
			while(top_p && pr(s[i]) <= pr(stk_p[top_p])) pop();
			stk_p[++ top_p] = s[i];
		}
	}
	while(top_p && pr(stk_p[top_p])) pop();
	if(top_p) puts("括号不匹配, 缺少右括号");
	else printf("%.6lf\n", stk_val[1]);
	
	return 0;
}
\end{lstlisting}

此代码已通过洛谷 \t{P10473 表达式计算4} 并利用下述数据进行测试.

$$
\begin{array}{|c|c|}\hline
	\t{输入} & \t{输出} \\\hline
	2^{\wedge}10-1000-(5^{\wedge} 3)+(3^{\wedge}2)  & -92 \\\hline
	1-2+3-4+5-6 & -3 \\\hline
	-(1-2^\wedge 3)^\wedge(-2)+1 & 0.979592 \\\hline
	2^\wedge 3*2/4^\wedge 2 & 1 \\\hline
	 2*(1+2^\wedge 2)/(3*2-3)*(1^\wedge 10+2)-3*2 & 4 \\\hline
	 ((1+2) & \t{缺少左括号} \\\hline
	 1+2) & \t{缺少右括号} \\\hline
\end{array}
$$

\subsection{股票走势}

\subsubsection{朴实的设计思想}

我们直接枚举一个右端点 $i$, 然后从 $i$ 开始往左枚举 $j$, 直到遇到第一个比当前值大的位置. 时间复杂度 $\t O(n^2)$.

\subsubsection{利用栈}

考虑维护一个单调栈, 此处我们需要找到左边第一个比自己大的位置, 那么就从左向右维护一个单调递减的单调栈.

具体过程就是, 从左至右遍历数组, 如果当前值比栈顶大, 就将栈顶弹出, 重复此过程. 那么栈顶元素就是左边最近的比当前值大的元素. 如果栈是空的, 就说明当前值左边没有比它大的, 即 $s_i = i$(当下标从 $1$ 开始). 在弹出之后, 将当前值入栈.

利用数学归纳法的思想不难证明, 按照上述操作维护的栈是单调的.

时间复杂度 $\t O(n)$.

具体代码如下.

\begin{lstlisting}
const int N=2e5+10;
int stk[N], top;
int n, a[N], s[N];
int main()
{
	read(n);
	for(int i = 1; i <= n; i ++)
	{
		read(a[i]);
		while(top && a[i] >= a[stk[top]]) top--;
		s[i] = i - stk[top];
		stk[++top] = i;
	}
	for(int i = 1; i <= n; i ++)
	write(s[i],i < n ? ' ' : '\n');
	flushout();
	return 0;
}
\end{lstlisting}

\section{Task 4: 基数排序}

\subsection{设计分析}

基数排序, 就是按数位进行排序, 而越高的数位, 其对排序的影响更大. 我们考虑从低位到高位依次排序.

对于每一位, 将所有数按照这一位一次加入到编号为 $0\sim 9$ 的十个队列中. 并按编号依次将这十个队列拼接, 每一位就基于上一位拼接的队列顺序进行分组.

而对于字符串同理, 只需将 $0\sim 9$ 改为 $a\sim z$ 用 26 个队列分组即可.

\subsection{复杂度分析}

在基数排序中, 我们可以认为整数就是由 $0\sim 9$ 构成的字符串. 所以在将前导 $0$ 补全后, 其复杂度与字符串一致.

记 $m$ 为数据的位数, $n$ 为待排序的数据个数. 则时间复杂度为 $\t O(mn)$.

\subsection{代码实现}

\subsubsection{整数}

\begin{lstlisting}
int main()
{
	freopen("radixSort1.txt","r",stdin);
	for(int v;scanf("%d",&v)!=EOF;)
	a.push(v);
	n = a.size();
	for(int p = 1, fg = 1; fg;p *= 10)
	{
		while(!a.empty())
		{
			int now = a.front();
			a.pop();
			q[now/p%10].push(now);
		}
		
		if(q[0].size() == n) fg = 0;
		
		for(int i = 0; i < 10; i ++)
		while(!q[i].empty())
		a.push(q[i].front()), q[i].pop();
	}
	while(!a.empty()) 
	res.push_back(a.front()),
	a.pop();
	return 0;
}

\end{lstlisting}

\subsubsection{字符串}

为了操作简单, 直接使用 \t{ASCII} 码进行字符映射,  所以 $q$  的编号范围在 $0\sim 127$ 但实际使用部分只有 52 个.

\begin{lstlisting}
int main()
{
	for(;scanf("%s",s[++ n])!=EOF;)
	a.push(n);
	n --;
	for(int p = 7;p >= 0; p--)
	{
		while(!a.empty())
		{
			int now = a.front();
			a.pop();
			q[s[now][p]].push(now);
		}
		
		for(int i = 0; i < 128; i ++)
		while(!q[i].empty())
		a.push(q[i].front()), q[i].pop();
	}
	while(!a.empty()) 
	res.push_back(a.front()),
	a.pop();
	return 0;
}
\end{lstlisting}



