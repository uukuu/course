
\makeatletter
\newcommand{\ifempty}[1]{\ifnum\pdf@strcmp{#1}{}=\z@}
\makeatother

\newcounter{mytheorem}[section]
\newcounter{mydef}[section]
\newcounter{myproblem}[section]
%\newcounter{special}[part]
%\counterwithin{chapter}{part}
%\numberwithin{figure}{chapter}
%\renewcommand{\thechapter}{第\zhnum{chapter}章}
%\renewcommand{\thesection}{\arabic{chapter}.\arabic{section}}
%\renewcommand{\thesubsection}{\arabic{chapter}.\arabic{section}.\arabic{subsection}}
%\renewcommand{\thesubsubsection}{\arabic{section}.\arabic{subsection}.\arabic{subsubsection}}
\renewcommand{\themydef}{\arabic{section}.\arabic{mydef}}
\renewcommand{\themytheorem}{\arabic{section}.\arabic{mytheorem}}
\renewcommand{\thefigure}{\arabic{chapter}.\arabic{figure}}

\numberwithin{equation}{chapter}
\renewcommand{\theequation}{\arabic{chapter}.\arabic{equation}}
\renewcommand{\t}[1]{\text {#1}}
\setitemize[1]{leftmargin=1.5cm}
%\theoremstyle{definition}
\newtheorem{theorem}[mytheorem]{\indent 定理}
\newtheorem{lemma}[mytheorem]{\indent 引理}
\newtheorem{proposition}[mytheorem]{\indent 命题}
\newtheorem{corollary}[mytheorem]{\indent 推论}
\newtheorem{definition}[mydef]{\indent 定义}
\newtheorem*{definition*}{\indent 定义}
\newtheorem{example}[mytheorem]{\indent 例}
\newtheorem{property}[mytheorem]{\indent 性质}
\newtheorem{remark}[mytheorem]{\indent 注}
\newenvironment{solution}{\begin{proof}[\indent\bf 解]}{\end{proof}}
\renewcommand{\proofname}{\indent\bf 证明}
\def\len{5pt}

%\tcbset{parbox=false,before upper=\par,before lower=\par}
\tcbset{breakable,before upper={\parindent2em},left=0pt,lefttitle=\len, right = 0pt, boxrule = 0.5mm, fonttitle = \bfseries}
\newenvironment{theorem}[1][]{
	\refstepcounter{mytheorem}
	\tcbset{title = {定理\themytheorem \ifempty{#1}\relax\else.(#1)\fi },colback=SeaGreen!10!CornflowerBlue!10,colframe=RoyalPurple!55!Aquamarine!100!}
	\begin{tcolorbox}
}{\end{tcolorbox}}

\newenvironment{lemma}[1][]{
	\refstepcounter{mytheorem}
	\tcbset{title = {引理\themytheorem \ifempty{#1}\relax\else.(#1)\fi },colback=Salmon!20,colframe=Salmon!90!Black}
	\noindent\begin{tcolorbox}
}{\end{tcolorbox}}


\newenvironment{corollary}[1][]{
	\refstepcounter{mytheorem}
	\tcbset{title = {推论\themytheorem \ifempty{#1}\relax\else.(#1)\fi },colback=Emerald!10,colframe=cyan!40!black}
	\noindent\begin{tcolorbox}
}{\end{tcolorbox}}

\newenvironment{proposition}[1][]{
	\refstepcounter{mytheorem}
	\tcbset{title = {},colback=white,colframe=white,colbacktitle=white,coltitle=black}
	\begin{tcolorbox}\noindent\hspace{\len}\textbf{命题\themytheorem.} \ifempty{#1}\relax\else(#1)\fi
}{\end{tcolorbox}}

\newenvironment{example}[1][]{
\refstepcounter{mytheorem}
\tcbset{title = {},colback=white,colframe=white,colbacktitle=white,coltitle=black}
\begin{tcolorbox}\noindent\hspace{\len}\textbf{例\themytheorem.} \ifempty{#1}\relax\else(#1)\fi
}{\end{tcolorbox}}

\newenvironment{property}[1][]{
\refstepcounter{mytheorem}
\tcbset{title = {},colback=white,colframe=white,colbacktitle=white,coltitle=black}
\begin{tcolorbox}\noindent\hspace{\len}\textbf{性质\themytheorem.} \ifempty{#1}\relax\else(#1)\fi
}{\end{tcolorbox}}

\newenvironment{remark}[1][]{
	\refstepcounter{mytheorem}
	\tcbset{title = {},colback=white,colframe=white,colbacktitle=white,coltitle=black}
	\begin{tcolorbox}\noindent\hspace{\len}\textbf{注\themytheorem.} \ifempty{#1}\relax\else(#1)\fi
	}{\end{tcolorbox}}

\newenvironment{remark*}[1][]{
	\tcbset{title = {},colback=white,colframe=white,colbacktitle=white,coltitle=black}
	\begin{tcolorbox}\noindent\hspace{\len}\textbf{注.} \ifempty{#1}\relax\else(#1)\fi
	}{\end{tcolorbox}}

\newenvironment{definition}[1][]{
	\setlength{\parindent}{2em}
	\refstepcounter{mydef}
	\tcbset{title = {定义\themydef\ifempty{#1}\relax\else.(#1)\fi },colback=OliveGreen!10,colframe=Green!70}
	\begin{tcolorbox}
}{\end{tcolorbox}}

\newenvironment{definition*}[1][]{
	\tcbset{title = {定义 \ifempty{#1}\relax\else(#1)\fi },colback=OliveGreen!10,colframe=Green!70}
	\begin{tcolorbox}
	}{\end{tcolorbox}}
\theoremstyle{definition}
%\newtheorem{theorem}[mytheorem]{\indent 定理}
%\newtheorem{lemma}[mytheorem]{\indent 引理}
%\newtheorem{proposition}[mytheorem]{\indent 命题}
%\newtheorem{corollary}[mytheorem]{\indent 推论}
%\newtheorem{definition}[mydef]{\indent 定义}
%\newtheorem*{definition*}{\indent 定义}
%\newtheorem{example}[mytheorem]{\indent 例}
%\newtheorem{property}[mytheorem]{\indent 性质}
%\newtheorem{remark}[mytheorem]{\indent 注}
\newenvironment{solution}{\begin{proof}[\indent\bf 解]}{\end{proof}}
\renewcommand{\proofname}{\indent\bf 证明}

\newenvironment{practice}{
	\practicetrue
	\section*{}
}{\practicefalse}

\newcommand{\problem}[1][]{
	\refstepcounter{myproblem}
	\ifempty{#1}
	\par\arabic{myproblem}.
	\else
	\par#1
	\fi
}

%字体颜色

\newcommand{\tr}[1]{\textcolor{red}{#1}}

%超链接
\newcommand{\hr}[1]{\hyperref[#1]{\t{#1}}}
\newcommand{\hrr}[2][]{\ifempty{#1}\hyperref[#2]{#2}\dotfill\pageref{#2}\else\hyperref[#1]{#2}\dotfill\pageref{#1}\fi}

\newcommand{\set}[2]{\mathbb {#1}^{#2}}
\newcommand{\seta}[2]{\mathbb {#1}_{#2}}
\newcommand{\setb}[3]{\mathbb {#1}^{#2}_{#3}}
\renewcommand{\t}[1]{\text {#1}}
\newcommand{\tb}[1]{\textbf{#1}}

\newcommand{\dps}[1]{\displaystyle {#1}}
\newcommand{\R}{\mathbb R}
\newcommand{\Z}{\mathbb Z}
\newcommand{\C}{\mathbb C}
\newcommand{\N}{\mathbb N}
\newcommand{\Q}{\mathbb Q}

\newcommand{\mydef}[2][]{\label{\ifempty{#1}#2\else #1\fi}\textbf{#2}}
%\newcommand{\mydef}[1]{\label{#1}\textbf{#1}}

%抽象代数

\newcommand{\Aut}{\t {Aut}}  %自同构群
\newcommand{\Inn}{\t {Inn}}  %内自同构群
\newcommand{\Ker}{\t {Ker}}  %核空间
\newcommand{\tIm}{\t {Im}}  %像空间

\newcommand{\Sy}{\t{Sylow} $p$-子群} %Sylow p-子群
\newcommand{\lAbel}{有限 \t{Abel}\ }
\newcommand{\Abel}{\t{Abel}\ }
\newcommand{\rad}{\t{rad}\ }

% 数学分析

\newcommand\mint[1][]{\dps\int_{\hspace{-0.4em} #1}}
\newcommand{\ti}{\t i}
\newcommand{\td}{\ \t d}
\newcommand{\fly}{\t{Fourier}\ }
\newcommand{\lpxc}{\t{Lipschitz}\ 条件}

%常微分方程
\newcommand{\wfen}[3][]{\dfrac{\td #2^{#1}}{\td^{#1} #3}}
\newcommand{\pfen}[3][]{\dfrac{\partial #2^{#1}}{\partial^{#1} #3}}
