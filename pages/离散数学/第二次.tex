\chapter{第二次作业}

\section{题目 P131/1}
设 $A=\{1,2,3\},B=\{a,b\}$, 求

(1)$A\times B$; (2)$B\times A$; (3)$B\times B$; (4)$2^B\times B$.

\begin{solution}\
	
	\begin{itemize}
		\item[(1)] $A\times B=\{(1,a),(1,b),(2,a),(2,b),(3,a),(3,b)\}$;
		\item[(2)] $B\times A=\{(a,1),(a,2),(a,3),(b,1),(b,2),(b,3)\}$;
		\item[(3)] $B\times B=\{(a,a),(a,b),(b,a),(b,b)\}$;
		\item[(4)] $2^B\times B=\{(\varnothing,a),(\varnothing,b),(\{a\},a),(\{a\},b),(\{b\},a),(\{b\},b),(\{a,b\},a),(\{a,b\},b)\}$.
	\end{itemize}
\end{solution}

\section{题目 P131/3}

证明 $(A\cap B)\times (C\cap D)=(A\times C)\cap (B\times D)$.

\begin{proof}\
	
$(a,c) \in (A\cap B)\times (C\cap D)\Leftrightarrow a\in A\cap B,c \in C\cap D\\ \Leftrightarrow (a,c) \in A\times C\wedge (a,c) \in B\times D \Leftrightarrow (a,c) \in (A\times C)\cap (B\times D)$
\end{proof}

\section{题目 P131/5}

设 $A=\{1,2,3\},B=\{a\}$, 求出所有 $A$ 到 $B$ 的二元关系.

\begin{solution}\
	
	由 $R\subseteq A\times B=\{(1,a),(2,a),(3,a)\}$
	
	得 $R:\varnothing,\{(1,a)\},\{(2,a)\},\{(3,a)\},\{(1,a),(2,a)\}, \\ \{(1,a),(3,a)\},\{(2,a),(3,a)\},\{(1,a),(2,a),(3,a)\}$.
\end{solution}

\section{题目 P132/6}

设 $A=\{1,2,3,4\},R_1=\{(1,3),(2,2),(3,4)\},R_2=\{(1,4),(2,3),(3,4)\}$. 求 $R_1\cup R_2,R_1\cap R_2,R_2\backslash R_1,R_2',\mathscr{D}(R_1),\mathscr{D}(R_2),\mathscr{R}(R_1),\mathscr{R}(R_2),\mathscr{D}(R_1\cup R_2),\mathscr{R}(R_1\cap R_2)$.

\begin{solution}\
	
	\begin{itemize}
		\item $R_1\cup R_2=\{(1,3),(2,2),(3,4),(1,4),(2,3)\}$;
		\item $R_1\cap R_2=\{(3,4)\}$;
		\item $R_2\backslash R_1=\{(1,4),(2,3)\}$;
		\item $R_2'=\{(1,1),(1,2),(1,3),(2,1),(2,2),(2,4),(3,1),(3,2),(3,3),(4,1),(4,2),(4,3),(4,4)\}$;
		\item $\mathscr{D}(R_1)=\{1,2,3\}$;
		\item $\mathscr{D}(R_2)=\{1,2,3\}$;
		\item $\mathscr{R}(R_1)=\{3,2,4\}$;
		\item $\mathscr{R}(R_2)=\{4,3\}$;
		\item $\mathscr{D}(R_1\cup R_2)=\{1,2,3\}$;
		\item $\mathscr{R}(R_1\cap R_2)=\{4\}$;
	\end{itemize}
\end{solution}

\section{题目 P132/7}

设 $R_1$ 和 $R_2$ 是从集合 $A$ 到 $B$ 得二元关系, 证明
\begin{itemize}[leftmargin=1.5cm]
	\item[(1)] $\mathscr{D}(R_1\cup R_2)=\mathscr{D}(R_1)\cup\mathscr{D}(R_2)$;
	\item[(2)] $\mathscr{R}(R_1 \cap R_2)\subseteq \mathscr{R}(R_1)\cap \mathscr{R}{R_2}$;
\end{itemize}

\begin{proof}\
	
	$a\in \mathscr{D}(R_1 \cup R_2) \Leftrightarrow \exists b \in B,s.t. (a,b) \in R_1\cup R_2\Leftrightarrow (a,b)\in R_1 \vee (a,b)\in R_2 \Leftrightarrow a\in\mathscr{D}(R_1)\cup\mathscr{D}(R_2)$
	
	$b\in \mathscr{R}(R_1\cap R_2) \Rightarrow \exists a\in A,s.t.(a,b) \in R_1\cap R_2 \Rightarrow (a,b) \in R_1 \wedge (a,b) \in R_2 \Rightarrow b \in \mathscr{R}(R_1)\wedge \mathscr{R}{R_2}\Rightarrow  \mathscr{R}(R_1 \cap R_2)\subseteq \mathscr{R}(R_1) \cap \mathscr{R}(R_2)$.
\end{proof}

\section{题目 P132/9}

定义在整数集合 $\mathbb Z$ 上得相等关系, '$\leqslant$' 关系, '<' 关系, 全域关系, 空关系, 是否具有表中所指的性质, 请用 Y 或 N 填表.


$
\begin{array}{|c|c|c|c|c|c|}
	\hline
	& \text{自反的} & \text{反自反的} & \text{对称的} & \text{反对称的} & \text{传递的} \\\hline
	
	\text{相等关系} & Y & N & Y & N & Y \\\hline
	\leqslant \text{关系}  & Y & N & N & N & Y \\\hline
	< \text{关系} & N & Y & N & Y & Y \\\hline
	\text{全域关系} & Y & N & Y & N & Y \\\hline
	\text{空关系} & N & Y & Y  & Y & Y \\\hline
\end{array}
$
