\chapter{第四次作业}

\section{习题五 3}

\begin{itemize}[leftmargin=1.5cm]
	\item[(1)] 单射
	\item[(3)] 都不是
	\item[(5)] 双射
	\item[(7)] 都不是
\end{itemize}


\section{习题五 6}

\begin{proof}
	$f\subseteq g\Rightarrow \mathscr{D}(f)\subseteq\mathscr{D}(g)\Rightarrow\mathscr{D}(f)=\mathscr{D}(g)$.
	
	从而 $\forall x\in \mathscr{D}(g),f(x)=g(x),\Rightarrow g\subseteq f,\Rightarrow g=f.$
\end{proof}

\section{习题五 9}

\begin{solution}
	\begin{itemize}
		\item[(1)]
		$$
		\begin{array}{l}
			f\circ g=f(g(x))=f(x+2)=(x+2)^2-1=x^2+4x+3,\\
			g\circ f=g(f(x))=g(x^2-1)=x^2+1.
		\end{array}$$
		\item[(2)] $g$ 是双射, $f,f\circ g,g\circ f$ 均不是.
	\end{itemize}
\end{solution}

\section{习题五 10}

\begin{solution}
	\begin{itemize}
		\item[(1)] 取 $f=(1234)$ 即 $f(1)=2,f(2)=3,f(3)=4,f(4)=1$.
		$$
		\begin{array}{c}
			f^2=(13)(24) \\
			f^3=(4321) \\
			f^{-1}=f^3 \\
			f\circ f^{-1}=I_A
		\end{array}
		$$
		\item[(2)] 取 $g=(12)(34)$, 即 $g(1)=2,g(2)=1,g(3)=4,g(4)=3$.
	\end{itemize}
\end{solution}

\section{习题五 11}
\begin{solution}
	$
	P^{-1}=\left(\begin{array}{c}
		123 \\
		231
	\end{array}\right)
	,\quad
	P\circ P^{-1}=\left(\begin{array}{c}
		123 \\
		123
	\end{array}\right)
	$
\end{solution}
\section{习题五 12}
\begin{solution}
	\begin{itemize}
		\item[(1)] 不是, 因为 $A\cap B \subseteq B$.
		\item[(2)] 是, 因为 $A \subseteq B\cup A$.
		\item[(3)] 是, 因为 $A$ 无穷那么存在可数集 $C\subseteq A$, 又 $C\backslash B$ 仍为可数集且 $C\backslash B \subseteq A\backslash B$.
	\end{itemize}
\end{solution}
