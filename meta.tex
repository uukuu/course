% meta.tex
% Copyright 2016 Zheng Xie <xie.zheng777@gmail.com>
% https://github.com/Tedxz/xjtuthesis-x
%
% This work may be distributed and/or modified under the
% conditions of the LaTeX Project Public License, either version 1.3
% of this license or (at your option) any later version.
% The latest version of this license is in
%   http://www.latex-project.org/lppl.txt
% and version 1.3 or later is part of all distributions of LaTeX
% version 2005/12/01 or later.
%
% This work has the LPPL maintenance status `maintained'.
%
% The Current Maintainer of this work is Zheng Xie.
%
% xjtuthesis-x is a Derived Work of xjtuthesis. The original maintainer of
% xjtuthesis is Weisi Dai (multiple1902 <multiple1902@gmail.com>),
% who published the project on https://code.google.com/p/xjtuthesis/ (no
% longer accessable). Currently, xjtuthesis is maintained by Aetf, and can
% be accessed on https://github.com/Aetf/xjtuthesis.
%
% xjtuthesis-x includes bug fixes, new features and a user guide.
% For detail, please refer to Readme.md.
%
% If you want to contribute to xjtuthesis-x or become the maintainer of
% xjtuthesis-x, please feel free to contact me.

% 标题,中文
\ctitle{课程资料整理}

% 作者,中文
\cauthor{数学强基 2301 刘欣楠}

% 学科,中文,本科生不需要
\csubject{}

% 导师姓名,中文
\csupervisor{李四}

% 关键词,中文。用全角分号「;」分割
% 研究生的应首先从《汉语主题词表》中摘选
\ckeywords{数学专业课、专业基础课、知识点、作业}

% 提交日期,本科生不需要
\cproddate{\the\year 年\the\month 月}

% 论文类型,中文,本科生不需要
% 从理论研究、应用基础、应用研究、研究报告、软件开发、设计报告、案例分析、调研报告、其它中选择
\ctype{}

% 论文标题,英文
\etitle{}

% 作者姓名,英文
\eauthor{}

% 学科,英文,本科生不需要
\esubject{}

% 导师姓名,英文
\esupervisor{}

% 关键词,英文。用半角分号和一个半角空格「; 」分割
\ekeywords{Wikipedia; Free encyclopedia; Winner; Good morning}

% 学科门类,英文
% 从Philosophy(哲学)、Economics(经济学)、Law(法学)、Education(教育学)、Arts(文学)、
%   Science(理学)、Engineering Science(工学)、Medicine(医学)、Management Science(管理学)中选择
\ecate{}

% 提交日期,英文,本科生不需要
% 应当和 cproddate 保持一致
\eproddate{\monthname{\month}\ \the\year}

% 论文类型,英文,本科生不需要
% 从Theoretical Research(理论研究)、Application Fundamentals(应用基础)、Applied Research(应用研究)、
%   Research Report(研究报告)、Software Development(软件开发)、Design Report(设计报告)、
%   Case Study(案例分析)、Investigation Report(调研报告)、其它(Other)中选择
\etype{}

% 摘要,中文。段间空行
\cabstract{
}

% 摘要,英文。段间空行
\eabstract{
}
